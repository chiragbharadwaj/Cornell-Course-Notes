\section*{Lecture 2}

\begin{itemize}
    \item We know some \textsf{OCaml} syntax; let's talk about the semantics and what these PL terms mean in general
    \item Five key aspects of learning ANY PL:
    \begin{enumerate}
        \item \textit{Syntax}: How to write compilable language constructs?
        \item \textit{Semantics}: What does program mean? (Type-checking, evaluation rules)
        \item \textit{Idioms}: What are typical language patterns/features used for expression?
        \item \textit{Libraries}: What facilities/utilities are part of ``standard" package?
        \item \textit{Tools}: What things exist to make life easier? (toplevel, GUI IDE, debugger)
    \end{enumerate}
    \item Breaking down learning into 5 steps (in this order specifically) makes learning PLs easier
    \item CS 3110 focuses on \textit{semantics} and \textit{idioms}--\textit{syntax} is boring, need to learn every time
    \item Don't complain about syntax, won't change very likely and is part of original designer's vision
    \item Semantics = metatool: facilitates learning other languages
    \item Idioms = expressing oneself: facilitates becoming a better programmer
    \item \textit{Libraries}, \textit{tools} = crucical, but keep udpating so no point getting fixated (except maybe \LaTeX)
    \item Recall: Expressions can be arbitrarily large since subexpressions can contain subexpressions, etc. (rec. def.)
    \item Every expression has both syntax and semantics (see below for more on semantics):
    \begin{itemize}
        \item Type-checking rules: Produce a type or fail with a message (invalid program)
        \item Evaluation rules: Produce a value or exception or infinite loop (ONLY if expression already type-checks)
    \end{itemize}
    \item Notice we already talked about some of this last time (how to know function application is valid, etc.)
    \item Values are \textit{terminal} expressions, i.e. they don't need more evaluation (e.g. \texttt{34:int}, \texttt{42.:float}, \texttt{false:boolean})
    \item Note: all values are expressions, not all expressions are values (types are not expressions though)
    \item Types $\neq$ values! e.g. \texttt{int} is not a value, \texttt{34} is -- in HBNF, \texttt{value ::= <type>} means INSTANCE of type
    \item Notice that \texttt{string} is not really a primitive type$\ldots$ achieved by chaining \texttt{char}s (HBNF: \texttt{token} vs. \texttt{token*})
    \item So: \texttt{"CS " \^{} "3110"} is an expression, but not a value (see evaluation below)
    \item \texttt{expr1 op expr2 --> value1 \^{} value2 --> string1 \^{} string2 --> "CS " \^{} "3110" --> "CS 3110"}
    \item Alternatively: \texttt{expr1 op expr2 --> "CS 3110"}
    \item First version: \textit{small-step semantics}; second version: \textit{big-step semantics} (\texttt{-->} is approximately ``evaluates to")
    \item \textsf{OCaml} \texttt{utop} displays big-step semantics, underlying evaluation is done with small-step semantics (we use small-step semantics exclusively in CS 3110 when we show the computations) 
    \item Challenge: Tracing expression back up the \textit{parse tree} to get definition (i.e \texttt{"CS 3110" <-- expr1 op expr2})
    \item Technically the evaluation above traces an \textit{abstract-syntax tree}, doesn't show everything: \texttt{token}/\texttt{digit}, etc.
    \item We will discuss lexing, parsing, and grammar trees later (if not, see CS 2112/CS 4120 notes)
    \item \textsf{OCaml}'s execution model is based on evaluation reduction (i.e. small-step semantics)
    \item We already discussed evluation of most commands/primitives and what types they give, except for \lstinline{let}-bindings
    \item Consider generalized snippet of \textsf{OCaml} code: \lstinline{let x = expr1:t1 in expr2:t2} (types shown explicitly here)
    \item From now on, for simplification, we will use \texttt{t=type}, \texttt{d=decl}, \texttt{v=value}, \texttt{e=expr} (all others explicitly named)
    \item Easier to read: \lstinline{let x = e1:t1 in e2:t2} (carries same meaning, simply substitute aforementioned names)
    \begin{itemize}
        \item Type-checking: this entire expression evaluates to the type \texttt{t2} (recall the main rule from recitation)
        \item Evaluation: Multi-step process (see outline below for a thorough analysis that covers the tricky parts)
        \begin{enumerate}
            \item Evaluate \texttt{e1 --> v1} (to a terminal value)
            \item Substitute \texttt{v1} for \texttt{e1} in \texttt{e1} and name the result \texttt{e2'}
            \item Evaluate \texttt{e2' --> v} (to another terminal value)
            \item Result: ``return" \texttt{v} (in \textsf{OCaml}, \texttt{return} call is optional: result of expression always returned)
            \item Note: Type of expression is type of \texttt{v}, which is type of \texttt{e2'}, which is type of \texttt{e2}, which is \texttt{t2}
        \end{enumerate}
        \item Chaining the process above helps evaluate complex sub-expressioned expressions (\texttt{e2' --> e3 --> e3'})
    \end{itemize}
    \item Bad idiom: Multiple variable bindings of the same name (``shadowing")
    \begin{lstlisting}
    let x = 6 in
        ((let x = 5 in x) + x) 
    \end{lstlisting}
    \item This becomes: \lstinline{let x = 6 in (5 + x) --> (5 + 6) --> 11} (we'll talk about this in a few lectures)
\end{itemize}