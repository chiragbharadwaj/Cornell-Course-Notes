\subsection*{Recitation 2}

\begin{itemize}
    \item Need to cover some minor features we may have missed, some details on evaluation and substitution model
    \item Note for below: we use \texttt{def ::= \textit{(* definition *)}} to make a definition, \texttt{def := <type>} to specify its type
    \item Some infix binary operators we missed:
    \begin{enumerate}
        \item Exponentiation: \texttt{** := float -> float -> float} (can customize a function to extend to \texttt{int}s as well)
        \item Logical \texttt{and}: \texttt{\&\& := boolean -> boolean -> boolean}
        \begin{itemize}
            \item In \texttt{a \&\& b}, if \texttt{a} is \texttt{false}, then \texttt{a \&\& b} short-circuits to \texttt{false}
        \end{itemize}
        \item Logical \texttt{or}: \texttt{|| := boolean -> boolean -> boolean}
        \begin{itemize}
            \item In \texttt{c || d}, if \texttt{c} is \texttt{true}, then \texttt{c || d} short-circuits to \texttt{true}
        \end{itemize}
        \item Bitwise \texttt{and}: \texttt{land := int -> int -> int}
        \item Bitwise \texttt{or}: \texttt{lor := int -> int -> int}
        \item Bitwise \texttt{xor}: \texttt{lxor := int -> int -> int}
        \item Bitwise \texttt{not}: \texttt{lnot := int -> int}
        \begin{itemize}
            \item This is included for completion; it is actually a unary prefix operator
        \end{itemize}
        \item Logical left shift: \texttt{lsl := int -> int -> int}
        \begin{itemize}
            \item The arithmetic left shift is the same as the logical one (see CS 3410 notes for more on digital logic)
        \end{itemize}
        \item Logical right shift: \texttt{lsr := int -> int -> int}
        \item Arithmetic right shift: \texttt{asr := int -> int -> int}
    \end{enumerate}
    \item Distinction between \textit{true equality} and \textit{Leibniz equality}: Former refers to same mathematical object while latter refers to structural equality, i.e. two objects are indiscernable property-wise but are distinct in allocation
    \item In practice: true equality is equality of memory address, while structural equailty is equality of properties/fields
    \item \textsf{OCaml}: There are separate equal/unequal operators for checking true vs. structural equality (\textsf{Java}-esque)
    \begin{itemize}
        \item True equality: \texttt{==} (equal), \texttt{!=} (unequal)
        \item Structural equality: \texttt{=} (equal), \texttt{<>} (unequal)
    \end{itemize}
    \item The other thing we missed were some simple floating-point mathematical functions that make life easier:
    \begin{enumerate}
        \item Absolute value: \texttt{abs := float -> float}
        \item Arc-cosine: \texttt{acos := float -> float}
        \item Arc-sine: \texttt{asin := float -> float}
        \item Arc-tangent: \texttt{atan := float -> float}
        \item Cosine: \texttt{cos := float -> float}
        \item Hyperbolic cosine: \texttt{cosh := float -> float}
        \item Exponential ($e$): \texttt{exp := float -> float}
        \item Natural logarithm: \texttt{log := float -> float}
        \item Common logarithm: \texttt{log10 := float -> float}
        \item Sine: \texttt{sin := float -> float}
        \item Hyperbolic sine: \texttt{sinh := float -> float}
        \item Square root: \texttt{sqrt := float -> float}
        \item Tangent: \texttt{tan := float -> float}
        \item Hyperbolic tangent: \texttt{tanh := float -> float}
    \end{enumerate}
    \item Challenge: Use math skills to derive all other needed functions and implement them into your own library
    \item Now to explore the \textit{substitution model} we've been using all along in our evaluations
    \item Quick notation simplification before we start using it: \texttt{e\{v/x\}} means ``evaluate \texttt{e} with \texttt{v} plugged in for \texttt{x}", or more precisely ``\lstinline{let x = v in e}" (this is the better intuition for evaluation)
    \item A quick example of how we evaluated using small-step semantics in our substitution model:
    \begin{lstlisting}
    let x = 3 in (x + x) --> (3 + 3) --> 6
    \end{lstlisting}
    \item There are generalized rules we must use, in case expressions themselves define new variables:
    \begin{enumerate}
        \item \texttt{n\{v'/x\} ::= n}
        \item \texttt{x\{v/x\} ::= v}
        \item \texttt{y\{v/x\} ::= y} if \texttt{y} is not equal to \texttt{x} (equality becomes Case 2)
        \item \texttt{(e1 op e2) \{v/x\} ::= e1\{v/x\} op e2\{v/x\}}
        \item \texttt{(let x = e1 in e2) \{v/x\} ::= let x = e1\{v/x\} in e2}
        \item \texttt{(let y = e1 in e2) \{v/x\} ::= let y = e1\{v/x\} in e2\{v/x\}} if \texttt{x} is not equal to \texttt{y}
    \end{enumerate}
    \item Let's move onto \textsf{OCaml} evaluation rules; recall that every \textsf{OCaml} program is an expression:
    \begin{enumerate}
        \item Double expression operated reduction:
        \begin{lstlisting}
        e1 op e2 --> e1' op e2
           (|if|) e1 --> e1'
        \end{lstlisting}
        \item Single expression operated reduction:
        \begin{lstlisting}
        v1 op e2 --> v1 op e2'
           (|if|) e2 --> e2'
        \end{lstlisting}
        \item Numerical operated reduction:
        \begin{lstlisting}
        n1 op n2 --> n3
               (|where|) n3 is the binop result (|of|) n1 (|and|) n2
        \end{lstlisting}
        \item Triple expression bound reduction:
        \begin{lstlisting}
        if e1 then e2 else e3 --> if e1' then e2 else e3
                        (|\hspace{2.5pt}if|) e1 --> e1'
        \end{lstlisting}
        \item True branched reduction:
        \begin{lstlisting}
        if true  then e2 else e3 --> e2
        \end{lstlisting}
        \item False branched reduction:
        \begin{lstlisting}
        if false then e2 else e3 --> e3
        \end{lstlisting}
        \item Double expression bound reduction:
        \begin{lstlisting}
        let x = e1 in e2 --> let x = e1' in e2
                   (|\hspace{1pt}if|) e1 --> e1'
        \end{lstlisting}
        \item Double expression reduction:
        \begin{lstlisting}
        e1 e2 --> e1' e2
        (|if|) e1 --> e1'
        \end{lstlisting}
        \item Single expression reduction:
        \begin{lstlisting}
         v e2 --> v e2'
        (|if|) e2 --> e2'
        \end{lstlisting}
    \end{enumerate}
    \item It may seem like there are other ones, but they can mostly be derived from these with the substitution model and expression-level chaining
    \item Once the entire program is fully-reduced to a terminal value, execution is completed (and the resulting displayed type on the toplevel, or REPL, may very well be extremely compounded and complex)
    \item We need to extend our HBNF definitions table now, since we've added a lot of extra terms (seems like it will never be complete)
    \item Will modify the HBNF definitions list after next lecture, once we cover more (i.e. alternative declarations, functions, etc.)
    \item Need to remember key point: Substitution model and evaluation rules are used in conjunction to analyze and reduce a program to a final value with a clear, discernable type (note, again, that this STILL allows for generic types, which have been determined to be such)
\end{itemize}
